\documentclass{article}

\usepackage{german}
\usepackage{hyperref}
\usepackage{amsmath}


\title{Projektseminar LoRa \\ \large Anbindung einer Wetterstation an das LoRaWAN}
\date{\today}
\author{
  Lenny Reitz\\
  \and
  Raphael Rezaii-Djafari\\
}

\begin{document}
  \maketitle
  \newpage
  \tableofcontents
  \newpage

  \section{Einleitung}

    Das Projekt wurde im Laufe des Projektseminars 2021 von Lenny Reitz und Raphael Rezaii-Djafari unter der Führung von Professor Jörg Vogt an der HTW Dresden entwickelt.

    \subsection{LoRa \& LoRaWAN}

    \subsection{Problemstellung}

    \subsection{Umsetzung}


  \section{Einrichtung}

    \subsection{Hardware}

      \begin{itemize}
        \item\href{https://learn.sparkfun.com/tutorials/weather-meter-hookup-guide#resources-and-going-further}{Wettermessgerät-Kit} SEN-15901
        
        \begin{itemize}
          \item beinhaltet Regenmesser, Anemometer, Windfahne
        \end{itemize}

        \item Breadboard
        \item Jumper-Kabel
        \item 10 kOhm Widerstand
        \item 2 \href{https://www.pollin.de/p/modular-einbaubuchse-mit-anschlusslitzen-6p6c-541843}{Modular-Einbaubuchsen mit Anschlusslitzen, 6P6C}
        \item Arduino Uno
        \item \href{https://wiki.dragino.com/index.php?title=Lora_Shield}{Dragino Lora Shield v1.4}
      \end{itemize}

      \subsubsection{Aufbau}

    \subsection{Software}

      \subsubsection{Libraries}
    
      TODO - LoRa Library

  \section{Funktionsweise}

    \subsection{LoRa}

    \subsection{Sensoren der Wetterstation}

      \subsubsection{Regenmesser}

        Der Regenmesser besteht aus einem Auffangbehälter und einem darin befindlichen einfachen Kippschalter.
        Wenn sich genügend Wasser im Behälter gesammelt hat, kippt der Schalter um und das Wasser läuft wieder aus dem Behälter heraus.
        Ein Kippvorgang entspricht einer Wassermenge von 0,2794 mm.
        Die folgende Umrechung ergibt die Wassermenge in Milliliter bei einer Auffangfläche von 55 cm²:

        \begin{multline}
          1\,mm = 1\,l/m^2 \quad \text{(Niederschlag)} \\
          \Rightarrow 0.2794\,mm = 0.2794\,l/m^2 = 0.02794\,ml/cm^2 \\
          \Rightarrow 0.02794\,ml/cm^2 * 55\,cm^2 \approx 1.54\,ml
        \end{multline}

        Ein Kippen beträgt also im Idealfall 1,54 ml. Mehrere Testläufe haben aber gezeigt, dass der Regenmesser mit einer relativ großer Ungenauigkeit schaltet.
        Je nach dem, wie schnell das Wasser einfließt, variiert die Messgenauigkeit stark.

      \subsubsection{Anemometer}

        Das Schalenanemometer misst Windgeschwindigkeiten durch Schließen des Kontaktes eines Schalters durch einen Magneten.
        Eine Windgeschwindigkeit von 2.4 km/h schließt den Schalter einmal pro Sekunde.

        Eine Umdrehung entspricht 3 Schaltvorgängen.

      \subsubsection{Windfahne}

        Die Windfahne besitzt 8 Schalter, welche jeweils mit unterschiedlich großen Widerständen verbunden sind.
        Die Schalter könne einzeln oder in Paaren umgelegt werden, wodurch sich 16 mögliche Positionen bestimmen lassen.

        Durch einen externen Widerstand wird ein Spannungsteiler implementiert, der eine messbare Spannung ausgibt.
        Daher lässt sich die Windrichtung nur durch ein analoges Signal bestimmen.

    \subsection{Beispielablauf}

      Benötigte Payload Größe (pro Sekunde):

      \begin{itemize}
        \item Windfahne: 4 Bit (16 mögliche Windrichtungen)
        \item Regenmesser: 2 Bit (maximal 4 Klicks pro Sekunde)
        \item Anemometer: 7 Bit (maximal 128 Klicks pro Sekunde)
      \end{itemize}

      => 13 Bit Payload Größe (pro Sekunde)
      Wenn wir 60 Messwerte pro Minute aufnehmen würden und dann pro Minute diese Daten an das TTN senden würden, bedeutet das:

      \begin{itemize}
        \item Payload Größe: 13 Bit * 60 = 780 Bit ~ 98 Byte (pro Minute)
      \end{itemize}

      Es muss eine Konfiguration für LoRa eingestellt werden, welche optimal für die zu sendende Paketgröße und Sendeinterval abgestimmt ist.

      Laut https://www.thethingsnetwork.org/airtime-calculator ist die minimale Airtime für unsere Paketgröße und Sendeinterval:

      \begin{itemize}
        \item Paketgröße: 98 Byte
        \item Spreading Factor: SF7
        \item Region: EU868
        \item Bandbreite: 250 kHz
      \end{itemize}

      => Airtime: 94,8 ms

      Problem, weil die Fair Use Policy von TTN nur 30s Airtime pro Tag erlaubt. Wir sind allerdings mit 137s pro Tag weit darüber.

      => Eine rohe Übermittlung der Daten mit einer Genaugikeit von 1s ist nicht möglich.

      Schlussfolgerung: Die Rohdaten müssen gemittelt werden, um eine Übertragung innerhalb der 30s pro Tag zu einzuhalten.

      
      pro 5 Sekunden:

      - Regenmesser: 3 Klicks * 5 Sekunden = 15 Klicks => 4 Bits
      - Anemometer: Wert wird gemittelt => 7 Bits
      - Windfahne: Wert wird gemittelt => 4 Bits

      => Gesamt: 15 Bits pro 5 Sekunden

      => Pro Minute: 180 Bits ~ 23 Byte


      pro 10 Sekunden:

      - Regenmesser: 3 Klicks * 10 Sekunden = 30 Klicks => 5 Bits
      - Anemometer: Wert wird gemittelt => 7 Bits
      - Anemometer: Maximalwert => 7 Bits
      - Windfahne: Wert wird gemittelt => 4 Bits

      => Gesamt: 23 Bits pro 10 Sekunden

      => Pro Minute: 138 Bits ~ 18 Byte

      => pro 5 Minuten: 690 Bits ~ 87 Byte

      Das wäre eine mögliche Konfiguration mit einer Airtime von 87.2 ms pro 5 Minuten oder 25.1 s pro Tag
      

        

\end{document}